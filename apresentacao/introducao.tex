\section{Introdução}

% ------ estrutura
%-introdução
%%-motivação
%%-objetivos
%%-

\subsection{}
    \begin{frame}\frametitle{Introdução}
        \begin{itemize}
            \item Disseminação de dispositivos com acesso à internet. % mais e mais pessoas acessam e disponibilizam dados e vídeos
            \item Evolução da tecnologia. % acesso a vídeos com cada vez mais qualidade
            \item Amplo acesso à vídeos digitais. % exemplo do youtube
        \end{itemize}
    \end{frame}

\subsection{}
    \begin{frame}\frametitle{Motivação}
        \begin{itemize}
            \item Inexistência de uma plataforma integrada para avaliações objetivas e subjetivas e degradação de vídeos.
            \item Auxílio no desenvolvimento de novas métricas e comparação com as existentes.
            \item .
        \end{itemize}
    \end{frame}

\subsection{}
    \begin{frame}\frametitle{Objetivos}
        \begin{itemize}
            \item Aprimorar os algoritmos de geração de artefatos de vídeo digital (blocagem e borramento). % de maneira a permitir a manipulação de parametros
            \item Adaptar a ferramenta SASQV para manipular vídeo bruto no formato .yuv.
            \item Adicionar à ferramenta um simulador de transmissões de vídeo via rede, simulando o \emph{streaming} e seus possíveis artefatos.
            \item Desenvolver uma nova interface gráfica em linguagem Java de funcionalidade similar à do SASQV.
            \item Aprimorar a portabilidade da ferramenta para sistemas operacionais diversos.
            \item Distribuir as ferramentas e o \emph{software} desenvolvido sob a forma de \emph{software livre}.
        \end{itemize}
    \end{frame}
    
\subsection{}
    \begin{frame}\frametitle{Justificativa}
        \begin{itemize}
            \item Integrar conhecimentos adquiridos ao longo do curso de Engenharia de Computação.
            \item Aprofundar conhecimentos sobre processamento de imagens.
            \item Trabalhar numa aplicação de necessidade real.
        \end{itemize}
    \end{frame} 
