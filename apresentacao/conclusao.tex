\section{Conclusão}

% ------ estrutura
\subsection{Conclusões}
    \begin{frame}\frametitle{Completude dos Objetivos}
	Quanto às ferramentas:
	\begin{itemize}
		\item Geradores de artefatos:
		\begin{itemize}
			\item Configuráveis de forma flexível.
			\item Oferecem mais opções.
			\item Trabalham sobre arquivos YUV.
		\end{itemize}
		\item{Arquivo de degradação}
		\begin{itemize}
			\item Permite reprodução e controle sobre os artefatos.
			\item Economia de espaço.
		\end{itemize}
		\item Simulador:
		\begin{itemize}
			\item Dispensa \emph{interfaces} físicas.
			\item Configurações flexíveis.
		\end{itemize}
	\end{itemize}
    \end{frame}

	\begin{frame}\frametitle{Completude dos Objetivos}
	Quanto à \emph{Interface}:
	\begin{itemize}
		\item Funcionalidades semelhantes ao SASQV.
		\item Adaptações para agregar ferramentas e novas utilidades.
	\end{itemize}
	Quanto ao sistema:
	\begin{itemize}
		\item Portável, disponível em um arquivo \emph{.jar}.
		\item Ferramentas independentes, disponíveis em C++.
		\item Documentos tutoriais de ajuda. % diferencial do SASQV original, que nao tinha nada
	\end{itemize}
	\end{frame}
    
\subsection{Trabalhos Futuros}
    \begin{frame}\frametitle{Sugestões}
	\begin{itemize}
		\item Extender a funcionalidade do \emph{hardware}.
		\item Novas ferramentas independentes podem ser criadas.
		\item Adaptações no BD, agregando mais resultados.
		\item Aprimoramento da simulação.
	\end{itemize}
    \end{frame}
    
\subsection{Fim}
    \begin{frame}\frametitle{Obrigado!}
        \begin{table}[!h]
	        \begin{tabular}{lll}
                Brunno Braga & 41-8898-4111 & brunnobga@gmail.com \\
                Caio Andreatta & 41-8448-7157 & caioccaa@gmail.com \\
	        \end{tabular}
        \end{table}
    \end{frame}
