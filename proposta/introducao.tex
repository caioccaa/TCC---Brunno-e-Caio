\chapter{Introdução}

Contando com os avanços tecnológicos das últimas décadas, o vídeo digital se encontra cada vez mais presente no dia a dia das pessoas ao redor do mundo, seja como educação, entretenimento ou informação.
Além de acessível, também é necessário que o vídeo chegue a essas pessoas com um nível de qualidade que as possibilitem contemplar e interpretar as imagens de forma natural, sem exasperação do (\sigla{SVH}{Sistema Visual Humano}, ou seja, sem que existam perdas ou distorções exageradas em seu conteúdo.
No entanto, processos como aquisição, compressão, armazenamento e transmissão, os quais viabilizam e popularizam a difusão de vídeo, são muitas vezes responsáveis por também introduzir artefados que degradam a qualidade final da imagem\cite{daronco}.

No intuito de melhor avaliar o efeito que tais artefatos podem produzir no SVH foram desenvolvidas métricas objetivas e subjetivas de avaliação de qualidade de vídeo. O projeto SASQV, desenvolvido por \cite{sasqv}, busca implementar ambas as formas de avaliação em um mesmo conjunto de \emph{software} e \emph{hardware} onde é possível gerar artefatos articiais, coletar avaliações subjetivas e aplicar métricas objetivas sobre uma base de vídeos, também oferencendo ferramentas para análise e comparação dos dados obtidos. O presente trabalho propõe adaptações ao SASQV, buscando aperfeiçoar a ferramenta de  geração de artefatos no sentido de torna-los mais similares àqueles encontrados nas trasmissões, fornecendo maior grau de liberdade para manipulação destes artefatos e também agregando novos modelos de artefatos, como por exemplo a simulação de um \emph{streaming} de vídeo. Dada a natureza do projeto SASQV, o qual se utiliza de softwares e bibliotecas distribuidos sob licenças de \emph{software} livre, este projeto também se propõe a construir uma nova interface gráfica não baseada em tecnologias proprietárias.

A proposta é motivada pela ampla difusão de vídeo digital, implantadas principalmente na forma de TV digital e de \emph{streaming} via internet, sendo o segundo objeto de grande interesse no mercado. Segundo \cite{sandvinereport}, no continente norte-americano cerca de 37\% do tráfego de internet fixa é dedicado ao \emph{streaming} de vídeo durante o horário onde tradicionalmente ocorre o pico de audiência televisiva, atingindo 41\% no caso da internet móvel dentro do mesmo horário.

\section{Palavras-Chave}
Vídeo digital, Geração de artefatos de vídeo digital, Avaliação subjetiva e objetiva de qualidade de vídeo.
\section{Objetivo Geral}
Adaptar a ferramenta \sigla{SASQV}{Sistema de Avaliação Subjetiva de Qualidade de Vídeo} no sentido de aprimorar a degradação, reprodução, manipulação, interação e portabilidade  presentes na ferramenta original.
\section{Objetivos Específicos}
\begin{itemize}
	\item \textbf{} Adaptar a ferramenta original para manipular vídeo bruto no formato \emph{.yuv}.
	\item \textbf{} Aprimorar os algoritmos de geração de artefatos de degradação de vídeo digital.
	\item \textbf{} Desenvolver uma nova interface gráfica em linguagem Java.
	\item \textbf{} Aprimorar a portabilidade da ferramenta para sistemas operacionais diversos.
\end{itemize}
