\chapter{Conclusão}

O resultado deste trabalho será uma ferramenta mais precisa e adequada para ser utilizada em meio acadêmico ou comercial, contribuíndo para o estudo e desenvolvimento da transmissão de imagens (por streaming, por meio televisivo, etc.). A ferramenta permitirá que novos trabalhos iniciem num ponto de partida mais avançado. 

Ainda, é possível evidenciar o conhecimento adquirido nas diversas disciplinas cursadas ao longo do curso de graduação em Engenharia de Computação, destacando-se aquelas onde foram abordadas técnicas de programação, gerenciamento de projetos, estrutura de dados, técnicas de processamento digital de imagens, entre outras. Observa-se a multidisciplinaridade e também o papel integrador deste trabalho.

Em sentido mais amplo, o produto deste trabalho pode vir a se tornar uma ferramenta comercial de avaliação subjetiva, podendo ter como objeto de avaliação tanto a qualidade da imagem quanto do conteúdo. Este tipo de ferramenta atinge um nicho de mercado ainda pouco explorado por transmissoras de sinal televisivo, mas que por outro lado já se mostra valioso quando explorado por distribuidoras de conteúdo audiovisual na internet.
