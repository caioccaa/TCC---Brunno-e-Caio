\chapter{Conclusão}

O resultado deste trabalho será uma ferramenta mais precisa e adequada para ser utilizada em meio acadêmico ou comercial, contribuíndo para o estudo e desenvolvimento da transmissão de imagens (por streaming, por meio televisivo, etc.). A ferramenta permitirá que novos trabalhos iniciem num ponto de partida mais avançado. 

É possível evidenciar neste trabalho o conhecimento adquirido nas diversas disciplinas cursadas ao longo do curso de graduação em Engenharia de Computação, destacando-se aquelas onde foram abordadas técnicas de programação, gerenciamento de projetos, estrutura de dados, técnicas de processamento digital de imagens, entre outras. Observa-se a multidisciplinaridade e também o papel integrador deste trabalho.

Em sentido mais amplo, este trabalho tem um caráter de retribuição à sociedade. Além de ser livre, a ferramenta visa contribuir para a melhora de um dos mais importantes meios de entretenimento contemporâneos.
